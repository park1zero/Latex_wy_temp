\documentclass[journal]{IEEEtran}

\usepackage{cite}
\usepackage{amsmath, amssymb, graphicx}
\usepackage{mathtools, dsfont}
\usepackage{array}
\usepackage[caption=false,font=normalsize,labelfont=sf,textfont=sf]{subfig}
\usepackage{url,kotex}

\newtheorem{theorem}{Theorem}    %% 정리
\newtheorem{definition}{Definition} %% 정의
\newtheorem{lemma}{Lemma}    %% 보조 정리
\newtheorem{proposition}{Proposition}    %% 보조 정리

\allowdisplaybreaks

\begin{document}

\title{DDP with CBF Constraints}

\author{Wonyoung Park (University of Seoul)\\ 
        Date: \today
        }

% Title
\maketitle

\section{본 문서의 소개}
본 문서는 Differential Dynamic Programming(이하 DDP)를 사용하여 자동차의 path planning 진행 사항에 대해 기록하고 있다. DDP를 이용하여 경로를 만들되, Control Barrier Function(이하 CBF)를 이용하여 constraint를 구성하고자 한다. 



\subsection{서울시립대학교 제어 및 동역학 연구실의 소개}


\begin{enumerate}
    \item CDSL@UoS는 서울시립대학교 전자전기컴퓨터공학부 소속으로, 수학적 모델에 기반한 제어이론(control theory)을 개발하고 이를 로보틱스(robotics)의 다양한 분야에 적용하는 연구를 수행하고 있습니다. 
    \item 본 연구실은 작성일 기준 박경훈 교수를 포함하여 8명의 연구자들로 구성되어 있습니다.
    \item 보다 자세한 사항은 연구실 홈페이지(\url{https://sites.google.com/view/cdsluos})를 참고해주세요. 
\end{enumerate}


\section{Prior Information}
\subsection{Differential Dynamic Programming}
DDP에 대해서 다룰 때, discrete-time system을 고려한다. 따라서 모델의 kinematics는 식\eqref{eq:state_space}와 같이 함수 f와 g로 구성된다.
\begin{align}
	x_{i+1}=f(x)+g(x)u
	\label{eq:state_space}
\end{align}

DDP의 total cost는
\begin{align}
	J(x_0,U)=\sum_{i=0}^{N-1} \ell(x_i,u_i)+\ell_f(x_N)
	\label{eq:total_cost}
\end{align}
식\eqref{eq:total_cost}와 같다. 여기서 $U \triangleq \{u_0,u_1,u_2...,u_{N-1}\}$이고, N은 총 길이, $ell_f$는 final cost를 의미한다.
이 상황에서 최소화를 하고자 하는 것은 $U^{\star}\triangleq\underset{U}{\arg\min} J(x_0,U)$이다.
i번째 경우일 때 Value는 $V_i(x)\triangleq \underset{U_i}{\min}J_i(x,U_i)$이다. 따라서 \eqref{eq:value}를 만족하게 된다.
\begin{align}
	 V(x)=\underset{u}{\min}[\ell(x,u)+V^{\prime}(f(x)+g(x)u)]
	\label{eq:value}
\end{align}
DDP는 iteration을 여러번 진행하며 \eqref{eq:value}를 backward pass에서 최소화하고, forward pass를 진행한다.

\subsubsection{Backward Pass}
\begin{align}
	Q(\delta x,\delta u) =\ell(x+\delta x,u+\delta u)\nonumber\\+V^{\prime}(f(x+\delta x)+g(x+\delta x)(u+\delta u))
	\label{eq:taylor_expansion}
\end{align}

\eqref{eq:value}를 Taylor expansion하면 \eqref{eq:taylor_expansion}과 같다. Q-function의 pseudo-Hamiltonian을 구해보면 다음과 같다.
\begin{align}
	Q_x&=\ell_x+f_x^{\top}V_x^{\prime} \nonumber\\
	Q_u&=\ell_u+f_u^{\top}V_x^{\prime} \nonumber\\
	Q_{xx}&=\ell_{xx}+f_x^{\top}V_{xx}^{\prime}f_x+V_x^{\prime}f_{xx}\nonumber\\
	Q_{ux}&=\ell_{ux}+f_u^{\top}V_{xx}f_x+V_x^{\prime}f_{ux}\nonumber\\
	Q_{uu}&=\ell_{uu}+f_u^{\top}V_{xx}^{\prime}f_u+V_x^{\prime}f_{uu}
	\label{eq:hamitonian}
\end{align}

Q에 대한 minimization을 진행하면
\begin{align}
 	\delta u ^{\star}=\underset{\delta u}{\arg\min} Q(\delta x,\delta u)=k+K\delta x
\end{align}
 가 만족된다.
추가적인 constarints가 존재하지 않는다면 locally-linear한 상황에서
\begin{align}
	k\triangleq-Q_{uu}^{-1}Q_u,
	K\triangleq-Q_{uu}^{-1}Q_{ux}
	\label{eq:kK}
\end{align}
를 만족한다.
k와 K를 구한 것을 이용하여 V에 대한 variation을 구할 수 있다.
\begin{align}
\Delta V&=-1/2k^{\top}Q_{uu}k\nonumber\\
V_x&=Q_x-K^{\top}Q_{uu}k\nonumber\\
V_{xx}&=Q_{xx}-K^{\top}Q_{uu}K
\label{eq:V_variation}
\end{align}

\subsubsection{Forward Pass}
Backward pass를 통해서 k,K의 값을 구하였고, 이를 이용하여 $\delta u^{\star}$를 구하였다.
이 $\delta u$를 통해서 u에 대한 update를 진행한다.

\begin{align}
	x_{0\_update}&=x_0\nonumber\\
	u_{i\_update}&=u_i+\delta u\nonumber\\
	x_{i+1\_update}&=f(x_{i\_update})+g(x_{i\_update})u_{i\_update}
	\label{eq:forward_pass}	
\end{align}
수식 \eqref{eq:forward_pass}에서의 $\delta u$는 $\alpha k_i+K_i(x_{i\_update}-x_i)$이다. $\alpha$는 1로 시작해서 점점 줄어드는 값이다.

\subsection{Control Barrier Function}
안전 영역 $\mathcal{C}$이 있다고 하고, 안전 영역의 구역을 다음과 같이 정의 가능하다.

\begin{align}
	\mathcal{C} \coloneqq \{x \in \mathds{R} : h(x) \geq 0 \}
	\label{eq:safe_region}
\end{align}

$h(x) \geq 0$을 만족하는 조건을 가지는 h(x)와 함께, 장벽 함수를 구현해야 한다.
크게 2가지의 장벽 함수를 고려했는데, 장벽 함수의 형태는 다음과 같다.
\begin{align}
	b_1=-log({h(x)\over{1+h(x)}}), b_2(x)={1\over{ h(x)}}
	\label{eq:barrier_function}
\end{align}
위와 같이 barrier function을 설정하면, h(x)가 안전 영역 $\mathcal{C}$안에 있는 경우에는 항상 양수의 값을 가진다. 다만, 안전 영역의 경계로 다가갈수록 함수의 값이 커지게 된다.\\
 \subsubsection{Reciprocal Barrier Function}
 \begin{align}
 	{1 \over \alpha_1(h(x)) \leq b(x) \leq {1\over \alpha_2(h(x))} }, L_f(b(x))\leq \alpha_3(h(x))	
	\label{eq:reciprocal_barrier_function} 
\end{align}
$\alpha$는 $class-\mathcal{K}$를 의미한다. 수식 \refeq{eq:reciprocal_barrier_function}은 reciprocal barrier function이라고 부른다.

\


\subsection{수식 작성}

기본적으로 LaTex에서 문서를 작성하는 mode는 크게 plain text를 작성하는 textmode와 수식을 작성하는 mathmode로 나뉩니다. 
문서 작성 시 textmode가 기본이며, mathmode로 진입하는 방법은 크게 아래의 두 경우입니다.  
\begin{itemize}
    \item 만약 plain text 내에서 수식을 작성하고 싶다면, ``\$$\cdots\text{(수식 내용)}\cdots$\$''와 같이 \$ symbol을 이용합니다. (예시 : $\dot{x} = Ax + Bu$)
    \item 만약 plain text와 plain text 사이에 단독으로 수식을 넣고 싶다면, {\tt align} 등의 함수를 이용하여 다음과 같이 작성합니다({\tt align} 외에도 {\tt equation}, {\tt aligned} 등의 함수가 있습니다.)
    \begin{align}\
        \dot{x} = Ax + Bu
    \end{align}
    이 때 {\tt align} 함수를 사용하면 위와 같이 수식 번호를 확인할 수 있는데, {\tt eqref} 함수를 이용하면 (식 \eqref{eq:State_Space}와 같이) 본문 상에서 수식 번호를 참조할 수 있습니다. 

    한편 수식 번호를 지우고 싶다면, $^*$ symbol을 추가하여 아래와 같이 작성하면 됩니다. 
    \begin{align*}
        y = Cx + Du
    \end{align*}

    또한 LaTex에서는 $\backslash\backslash$를 이용하여 줄바꿈을 하게 되는데, 이는 mathmode와 textmode에 공통으로 적용됩니다. 
    이와 함께 \&를 이용하면, 여러 줄로 구성된 수식을 가지런히 정렬한 형태로 작성할 수 있습니다.
    \begin{align}
        \dot{x} & = A x + B u,\\
        y & = C x + D u.
    \end{align}

\end{itemize}
 


\subsection{Theorem 환경 사용}

연구실에서 수행하는 연구로 논문을 작성할 때, 경우에 따라 definition, lemma, theorem 등을 문서에 추가할 필요가 있습니다. 
이 경우 IEEE 템플릿에서 공식 제공하거나 {\tt amsthm} package를 통해 {\tt newtheorem}으로 정의된 definition, lemma, theorem 함수 등을 아래와 같이 활용할 수 있습니다. 

\begin{definition}
    Blabla... $\hfill\square$
\end{definition}

\begin{lemma}
    Blabla... $\hfill\square$
\end{lemma}

\begin{theorem}\label{thm:Dummy}
Blabla... $\hfill\square$
\end{theorem}
\begin{IEEEproof}
    Blabla...
\end{IEEEproof}

수식과 마찬가지로 {\tt label} 함수를 이용하여 Theorem~\ref{thm:Dummy}과 같이 상호 참조도 가능합니다.

\subsection{Reference 작성}

LaTex에서는 {\tt thebibliography}라는 함수 혹은 {\tt bib} 파일을 이용하여 reference를 관리합니다. 
Reference 작성법은 별도로 연구실 Wiki를 통해 소개하고, 추후 템플릿에도 요약하여 정리해놓도록 하겠습니다. 
Rerefenrece 작성 예시는 본 문서의 제일 마지막을 참고해주세요. 

\subsection{유용한 참고 자료}

아래 리스트에서는 LaTex 사용과 관련하여 유용한 참고 자료들을 소개합니다. 
\begin{itemize}
\item {\bf Overleaf documentation} : \url{https://www.overleaf.com/learn}
\item {\bf LaTex 관련 Youtube 영상} (KAIST 수리과학과) : {\url{https://www.youtube.com/watch?v=ysnKC1jEC1s}}
\item {\bf Detexify} : \url{https://detexify.kirelabs.org/classify.html}
\item {\bf KTUG} (한국 텍 사용자 그룹) : 
\item {\bf LaTex - Beginner's Guide} : {\url{http://static.latexstudio.net/wp-content/uploads/2015/03/LaTeX_Beginners_Guide.pdf}}
\item {\bf IEEE Transactions Template 사용법 }: \url{https://www.cs.cmu.edu/~steffan/personal/tmp/IEEEtran_HOWTO.pdf}
\item (이후 업데이트 될 예정입니다.)
\end{itemize}



\appendices

\begin{thebibliography}{1}

\bibitem{cite:IEEE_Template}
\emph{IEEE Journal Paper Template}, IEEE, \url{https://www.overleaf.com/latex/templates/ieee-journal-paper-template/jbbbdkztwxrd}

\bibitem{cite:LaTex_Wiki}
\emph{LaTex}, Wikipedia, \url{https://en.wikipedia.org/wiki/LaTeX}

\end{thebibliography}

\end{document}